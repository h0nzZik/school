\documentclass[12pt]{ib102}
\usepackage[czech]{babel}

\setassignment{1}
\setexercise{2}
\setduedate{23}{9}{2013}
\setpoints{2}


%%% ZDE NAPISTE SVOJE UCO, JMENO A SKUPINU
\setstudent{Jan Tušil}
\setgroup{11}
\setuco{410062}


\begin{document}
\begin{zadani}
Nechť $L$ je jazyk nad abecedou $\Sigma=\set{a,b}$ tvořený právě všemi slovy,
která mají počet znaků $a$ nedělitelný $3$ a zároveň se nevyskytují 2 znaky $b$
za sebou (tedy mezi každými dvěma výskyty znaku $b$ je alespoň jeden znak $a$).


Zapište jazyk $L$ pomocí jednoprvkových jazyků $\set{a}$ a $\set{b}$
s~využitím konečného počtu operací sjednocení ($\cup$), průniku ($\cap$),
rozdílu ($\setminus$), doplňku ($\mathsf{co-}$), zřetězení ($\cdot$),
mocniny (${}^2, {}^3, \ldots$), iterace (${}^*$) a pozitivní iterace (${}^+$).

\end{zadani}

\begin{reseni}
Pro zpřehlednění zápisu si označme $A=\{a\}$, $B=\{b\}$, $U=A \cup (A.B)$.
Největší podmnožina jazyka $L$, ve které žádné slovo nezačíná písmenem $b$, může být zapsána:



\begin{math}
M=(U \cup U^2).(U^3)^*
\end{math}



Z té pak lze snadno utvořit jazyk $L$:


\begin{math}
L=M \cup (B.M)
\end{math}
\end{reseni}

\end{document}
