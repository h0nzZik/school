\documentclass[12pt]{ib102}
\usepackage[czech]{babel}

\setassignment{2}
\setexercise{2}
\setduedate{30}{9}{2013}
\setpoints{2}


%%% ZDE NAPISTE SVOJE UCO, JMENO A SKUPINU
\setstudent{Jan Tušil}
\setgroup{11}
\setuco{410062}


\begin{document}
\begin{zadani}
Uvažme jazyk
$$L = \{w \in \{a,b,c\}^*  \mid \#_a(w) + \#_b(w) = \#_c(w) \}.$$

\bigskip

\bigskip

\noindent Rozhodněte, zda je jazyk $L$ regulární, a vaše tvrzení dokažte. Tzn.:
\begin{itemize}
\item Pokud $L$ je regulární, uveďte regulární gramatiku, která $L$ generuje, nebo konečný deterministický automat, který $L$ akceptuje. Gramatiku/automat zapište se všemi formálními náležitostmi.
\item Pokud $L$ není regulární, dokažte tuto skutečnost pomocí Lemmatu o~vkládání (Pumping lemma).
\end{itemize}
\end{zadani}

\begin{reseni}
Slovo \( w = a^n.b^n.c^{2n} \) jistě leží v \( L \) pro každé \( n \in \mathbb{N^+} \). Rozdělení \(r.s.t=w \) taková, že \( \lvert s \rvert \geq 1 \land \lvert r.s \rvert \leq n \) lze vyjádřit výrazem
\begin{equation}
(a^{n-l-m}).((a^l)^i).(a^m.b^n.c^{2n})
\end{equation}
pro všechna celá \(l \geq 1 \land n-l-m \geq 0 \) a pevné \( i = 1 \). Slovo vzniklé dosazením \( i \geq 2 \) zjevně neleží v \( L \) (pouze se zvýší počet výskytu znaku \( a \) ), odtud z Pumping Lemma plyne, že \( L \) není regulární. 
\end{reseni}




\end{document}
